\documentclass[a4paper]{article}
%\documentclass[a4paper, twocolumn]{article}

% Options possibles : 10pt, 11pt, 12pt (taille de la fonte)
%                     oneside, twoside (recto simple, recto-verso)
%                     draft, final (stade de développement)
%\usepackage{graphicx}
\usepackage[pdftex]{graphicx}
\usepackage[utf8]{inputenc}
\usepackage[T1]{fontenc}
\usepackage[francais]{babel} 
\usepackage{fancyhdr}

\usepackage[a4paper]{geometry}
\usepackage{subfig}

\title{Calcul de couture minimale \\ {\large ESIR3 -- Imagerie Numérique -- CAV}}         
\author{Maxime \textsc{Broy} \and Florent \textsc{Guiotte}}
%\date{}                   
\pagestyle{fancy}

%\sloppy
\begin{document}
\maketitle 
\begin{abstract}
L'objectif initial du projet était l'implémentation du papier Scene Completion Using Millions of Photographs [Hays et Efros. 2007]. 
Ce dernier consiste à remplacer dans une image toute une région par une scène présente dans une autre image provenant d'une collection de photographies. 
L'algorithme explore cette collection jusqu'à trouver une image avec une sémantique correspondante sans couture.
On cherche donc à remplacer une région de l'image cible par quelque chose de <<plausible>>. 
Il s'agit d'une technique difficilement quantifiable puisque le résultat repose sur la perception visuelle humaine. 
En pratique, ce projet s'est concentré sur une des briques de cette algorithme~: 
le calcul de la couture minimale entourant l'objet que l'on veut transférer d'une image à une autre. 
\end{abstract}

\tableofcontents

\section{Introduction et concept}               % Commencer une section, etc.

\begin{figure}[!ht]%htp]
    \centering
    \subfloat[image cible]{\label{pres:bg}\includegraphics[width=0.48\textwidth]{img/bg.jpg}}
    \hspace{0.030\textwidth}
    \subfloat[image source]{\label{pres:fg}\includegraphics[width=0.48\textwidth]{img/fg.jpg}}
    \caption{Les images de la démonstration}
    \label{pres}
\end{figure}

L'idée est de <<découper>> une région dans l'image source, ici le voilier de la figure \ref{pres:fg} page
\pageref{pres:fg}, pour la transposer dans l'image cible figure \ref{pres:bg}. 
Le voilier devra alors s'intégrer le plus parfaitement possible dans son environnement. L'idée est donc de
calculer quels sont les pixels entourant le voilier que nous allons transposer avec lui dans l'image cible.

\section{Méthode}         
Notre méthode consiste à obtenir un masque de l'image source (figure \ref{gc:mask}) en utilisant la fonction GrabCut d'OpenCV. 
Ensuite, nous calculons l'énergie entre les deux images. C'est en utilisant le masque en sortie du GrabCut et l'énergie que nous 
pouvons ensuite calculer les cartes d'énergie cumulée. La couture qui minimise cette énergie cumulée sera celle retenue.

\subsection{Grabcut}      
On utilise la fonction GrabCut d'Open CV.
\begin{figure}[!ht]%htp]
    \centering
    \subfloat[image source]{\label{gc:fg}\includegraphics[width=0.48\textwidth]{img/fg.jpg}}
    \hspace{0.030\textwidth}
    \subfloat[masque]{\label{gc:mask}\includegraphics[width=0.48\textwidth]{img/mask.jpg}}
    \caption{Utilisation du GrabCut pour déterminer automatiquement un masque}
    \label{gc}
\end{figure}

L'algorithme du grabcut permet d'extraire le fond d'une image. 
Nous avons utilisé cet algorithme pour créer semi automatiquement le masque que nous utiliserons plus tard
pour déterminer le chemin de la couture minimum. 
Ce masque (figure \ref{gc:mask}) contient trois valeurs différentes~:

\begin{itemize}
    \item Le fond (noir), déterminé par l'utilisateur. 
    \item Le sujet (blanc), détouré par l'algorithme du grabcut selon les entrées de l'utilisateur.
    \item La marge (gris), correspond à la zone entre le fond et le sujet, ayant une probabilité d'appartenir au fond. C'est dans cette zone que nous allons chercher la couture minimale.
\end{itemize}

\subsection{Calcul d'énergie}

Nous avons essayé plusieurs techniques pour ce calcul d'énergie : Différence au carrée entre les deux images, Sobel en RBG ou LAB.
Voici le résultat pour une énergie calculée à partir de la différence au carrée :
\begin{figure}[!ht]%htp]
    \centering
    \subfloat[Valeur absolue de la différence au carré]{\label{energie:e}\includegraphics[width=0.48\textwidth]{img/energy.png}}
    \hspace{0.030\textwidth}
    \subfloat[Sobel]{\label{energie:sobel}\includegraphics[width=0.48\textwidth]{img/energy_sobel.png}}
    \caption{Carte d'énergie}
    \label{energie}
\end{figure}

\subsection{Calcul des cartes d'énergie cumulée}
\begin{figure}[!ht]%htp]
    \centering
    \subfloat[Valeur absolue de la différence au carré]{\label{ecum:e}\includegraphics[width=0.48\textwidth]{img/energy_cum.png}}
    \hspace{0.030\textwidth}
    \subfloat[Sobel]{\label{ecum:sobel}\includegraphics[width=0.48\textwidth]{img/energy_cum_sobel.png}}
    \caption{Cartes d'énergie cumulée}
    \label{ecum}
\end{figure}

Tout d'abord, il faut définir une ligne de pixels partant du bord du masque jusqu'à l’objet (on appellera cette ligne steam, pour start seam). Il s'agit de la ligne rouge sur la Figure 6.
Il y a une carte d’énergie par pixels de la steam.
Une carte d’énergie correspond à l’énergie cumulée de chaque pixel, en partant d’un pixel de steam jusqu’à faire le tour de l’objet en explorant le voisinage (connexité 8)

\subsection{Calcul de la couture}

\begin{figure}[!ht]%htp]
    \centering
    \subfloat[Valeur absolue de la différence au carré]{\label{allseams:e}\includegraphics[width=0.48\textwidth]{img/all_seams.png}}
    \hspace{0.030\textwidth}
    \subfloat[Sobel]{\label{allseams:sobel}\includegraphics[width=0.48\textwidth]{img/all_seam_sobel.png}}
    \caption{Coutures minimisant l'énergie, affichées sur carte d'énergie}
    \label{allseams}
\end{figure}

\begin{figure}[!ht]%htp]
    \centering
    \subfloat[Valeur absolue de la différence au carré]{\label{bestseam:e}\includegraphics[width=0.48\textwidth]{img/min_seam.png}}
    \hspace{0.030\textwidth}
    \subfloat[Sobel]{\label{bestseam:sobel}\includegraphics[width=0.48\textwidth]{img/min_seam_sobel.png}}
    \caption{Meilleure des coutures, affichée sur carte d'énergie cumulée}
    \label{bestseam}
\end{figure}
Pour chaque pixel qui n’est pas noir et qui n’est pas déjà visité, on explore ses voisins en sélectionnant celui dont l’énergie correspondante (dans la carte d’énergie) est minimale. Cela jusqu'à avoir atteint le point d'arrivée (voisin de droite du point de départ).

\section{Résultat}

Quisque dolor odio, aliquam quis, placerat sed, hendrerit eu, magna. Cras at
turpis et mi imperdiet lobortis. Nam eu massa et eros congue gravida. Sed
luctus. Nullam sit amet nunc a tellus lacinia tempor. Praesent tincidunt ligula
quis lacus. Nullam sodales, mi sed venenatis egestas, risus turpis dictum elit,
ac egestas augue eros eget erat. Cras faucibus.

\begin{figure}[!ht]%htp]
    \centering
    \subfloat[Valeur absolue de la différence au carré]{\label{results:e}\includegraphics[width=0.48\textwidth]{img/result.png}}
    \hspace{0.030\textwidth}
    \subfloat[Sobel]{\label{results:sobel}\includegraphics[width=0.48\textwidth]{img/result_sobel.png}}
    \caption{Fusion des deux images suivant la couture minimale}
    \label{results}
\end{figure}

\section{Performance}

Quisque dolor odio, aliquam quis, placerat sed, hendrerit eu, magna. Cras at
turpis et mi imperdiet lobortis. Nam eu massa et eros congue gravida. Sed
luctus. Nullam sit amet nunc a tellus lacinia tempor. Praesent tincidunt ligula
quis lacus. Nullam sodales, mi sed venenatis egestas, risus turpis dictum elit,
ac egestas augue eros eget erat. Cras faucibus.

\section{Références}

[1] J. Hays and A. A. Efros.  2007. Scene completion using millions of photographs. ACM Trans. Graph.,
\\

[2] JIA, J., SUN, J., TANG, C.-K., AND SHUM, H.-Y. 2006. Drag and-drop pasting. ACM Trans. Graph.,



\end{document}
